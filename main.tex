
\title{S{\"a}kerheten i webramverk f{\"o}r Erlang och Elixir}
\author{
		Lucas Persson\\  
		\href{mailto:lucaspe@kth.se}{lucaspe@kth.se} 
		  \and
		Johan Ejdemark\\  
		\href{mailto:ejdemark@kth.se}{ejdemark@kth.se}
		}
\date{\today}
 
\documentclass[12pt]{article}
\usepackage[T1]{fontenc}	      
%\usepackage[swedish]{babel}
\usepackage[english]{babel}
\usepackage[utf8x]{inputenc}
\usepackage{hyperref}
\usepackage{titlesec}
\usepackage[bottom]{footmisc}

\newcommand{\sectionbreak}{\clearpage}

\begin{document}
\maketitle

	 	 	 	 	
	
\section*{Sammanfattning}
%Förklarar problemet, varför det är ett problem, resultat och slutsats. Jag tycker inte att ”svaret” som ges i sammanfattningen och slutsatsen hänger ihop med frågan som presenteras i sammanfattningen och introduktionen.
Frågan var vilken säkerhet som är implementerad i de olika webbramverken för Erlang/Elixir samt vilka egna säkerhetsåtgärder som bör vidtas jämfört mot andra plattformer.
Om en attakerare kommer in på webservern så har den tillgång till hela applikations systemet samt all kommunikation.
Erlangs Virtuella Maskin är en stabil samt flertrådad VM, men har inte designats med säkerhet i åtanke, Erlang VM innehåller specifika säkerhetshål som går att täta till.
Undersökningen resulterade i insikten att webbramverken för Erlang och Elixir är till och med säkrare än deras motsvarigheter för objektorienterade språk.

\subsection*{Nyckelord}
Informationsflödeskontroll, Erlang, Elixir, Säkerhet, Webramverk, DDOS
	
\section{Inledning}

Frågeställningen var hur mycket säkerhet som är implementerad i de olika webbramverken\footnote{fördefinierad kombination med externa paket skrivna att passa med varandra som slutligen underlättar för utvecklare av webapplikationer} för Erlang\footnote{språket Erlang/OTP\cite{erlang}}/Elixir\footnote{dialekten av Erlang\cite{elixir}} samt vilka egna säkerhetsåtgärder som bör vidtas.
Detta är viktigt med hänseende på all känslig information som delges då man använder sig av webbtjänster.
%%Inledning bör innehålla en djupare förklaring av exakt vilka problemen var.

Av de vanligare hoten som finns mot webservrar så är den mest relevanta gällande Erlang/Elixir att säkerhetsregler inte är konfigurerade korrekt och på så sätt låter en attackerare komma åt Erlang VM\footnote{Erlang VM, Erlang BEAM virtuella maskin, kompilerar och kör Erlang/Elixir kod\cite{erlang}.} och exekverar skadlig kod. 

Sociala medier på webben hanterar personlig data från ett stort antal aktiva användare.
Dessa webbapplikationer måste följa en mängd olika krav på integritet, uppnå skalabilitet samt hög prestanda.
Att implementera begränsningar på flödet av data under användningen av dessa tjänster för att följa en viss integritetspolicy är en utmaning eftersom enskilda komponenter bearbetar data som tillhör många olika användare.

I en simpel blogg där flera användare skriver till en och samma anonyma diskussionstråd, kan det vara lämpligt att hindra var och en från att se den andres identitet.
I de bästa av världar, borde alla webbapplikationer implementera integritetspolicys genom att isolera data som tillhör olika användare.



 \section{Bakgrund och teori}

\subsection{Bakgrund}

\subsubsection*{Integritet}


Forskning om IFC\footnote{Med IFC så menas Information Flow Control eller informationsflödeskontroll} har visat att en möjlig lösning för att skydda integriteten är att kontrollera spridningen av data genom ett system.
Med IFC spårar systemet alla spår av datat i komponenterna samt hindrar oönskade dataflöden som skulle kunna skada integriteten.
Detta ger två problem:
Dels ger det mycket overhead\footnote{indirekt överflödigt processande av data} och dels blir det svårt att förverkliga.
Detta förhindrar IFC:n från att garantera isolering eftersom data som tillhör olika användare är blandat inom komponenten.
Det som är fördelaktigt med Erlang/Elixir i det här sammanhanget är att IFC:s restriktioner enbart måste bli kontrollerade då meddelanden skickas mellan processer.
 \footnote{sid 1\cite{IFC}}
\subsubsection*{Allmänt}
	För att skriva en säker Erlang/Elixir webserver krävs samma säkerhetsprinciper som vid objektorienterad serverutveckling, skydd mot Injection, Broken Authentication and Session Management, Cross-Site Scripting (XSS), Insecure Direct Object References, Security Misconfiguration, Sensitive Data Exposure, Missing Function Level Access Control, Cross-Site Request Forgery (CSRF), Using Components with Known Vulnerabilities, Unvalidated Redirects and Forwards.\cite{owasp}	



\begin{minipage}{\textwidth}
För att ha säker kod så behövs följande punkter implementeras:

\begin{itemize} 
		\item Kryptera kommunikationen mellan noder.\footnotemark
		\item Noder ska ha minimalt med privilegier.
		\item Implementera välkända och öppna krypterings paket.
		\item Använd alltid absoluta vägar till systemanrop.
		\item Öppna inte eller exponera mer funktioner än nödvändigt.
		\item Använd alltid de senaste stabila externa paketen.
	\end{itemize}
Dessutom kan användandet av kodanalysprogram visa säkerhetshål som en utvecklare inte tänker på.\cite{101}
\end{minipage}
\footnotetext{Med nod i denna rapport så menas en Erlang VM i ett skalbart nätverk som kör samma applikation samt är sammankopplade med andra noder}




\subsection{Teori}
Med integritet inom en blogg menas:

Meddelanden från en viss utgivare ska enbart tas emot av prenumeranter som denna har godkänt.
Detta försäkrar att utgivaren kan bibehålla kontrollen över vilka som tar emot meddelandena.

Godkända prenumeranter ska inte bli exponerade inför någon annan utgivare eller prenumerant.
Detta innebär att varje prenumeration görs privat samt att ingen annan användare av systemet, förutom den utgivare som godkänner prenumeranten, kan se vem som prenumererar.

Förfrågningar om prenumeration från eventuella prenumeranter ska enbart levereras till den utgivare som förfrågningen riktar sig till.
\footnote{sid 2\cite{IFC}}



\section{Metod och 	resultat }	



\subsection{ Metod}

Den valda metoden har varit litteraturstudie, källorna har bland annat funnits via databaserna IEEE Xplore  \footnote{ \url{ieeexplore.ieee.org}},  som analyserat olika rapporter och blogginlägg om webutveckling i erlang och i elixir, med hjälp av olika ramverk samt vilka säkerhetsaspekter som behövs ta hänsyn till.

För att rapporten ska hålla sig saklig var endast ramverken yaws\footnote{web servern Yet Another WebServer\cite{yawsorg}} (erlang), cowboy\footnote{ramverket Cowboy\cite{cowboyorg}} (erlang) och phoenix\footnote{webramverket Phoenix\cite{phoenixorg}} (elixir) analyserade.

Då inga rapporter gick att hitta rörande elixir har blogginlägg använts där författarna bedöms som opartiska till fakta.

Vid eftersökning av källor som kopplas till denna rapport hänvisades det till fyra inlägg i en viss spalt.
Samtliga fyra inlägg i spalten The Functional Web var dock skrivna av Steve Vinoski. På grund av att författaren utvecklar YAWS blir han partisk och hans allmänna påverkan på vad som återgivits i spalten kan ses som missvisande eller partisk, därför har spalterna inte innefattats i denna rapport.

\subsection{ Resultat}

\subsubsection*{Erlang VM}
	Erlang är snabbt samt enkelt att skala upp. Så man kan utnyttja andra säkerhetslösningar som skulle ta för lång tid i de vanligaste objektorienterade språk såsom att ta emot hela block från databasen och söka igenom dem nära slutanvändaren. Detta för att bibehålla vad användaren söker efter samt då eliminera möjligheten att analysera fram användarens syfte och kanske identitet\cite{database}.

Eftersom Erlang är så snabbt och skalbart kan det klara av DDOS\footnote{Med DDOS så menas Distributed Denial-of-Service attack, på svenska distribuerad förnekande av tjänst, vilken innebär att ett nätverk av attakerare skickar mer trafik till applikationen än vad applikationen klarar av att hantera} attacker mycket bättre/längre.\cite{resource-safe}

\subsubsection*{Ramverken}
YAWS är egentligen mer en statisk webserver med stöd för dynamiska sidor mer än ett ramverk. Hänsyn borde dessutom tas till samma saker som med webservrarna Apache HTTPD eller nginx i kombination med språket PHP, som att konfigurera vilka sidor och platser som får besökas samt vilka platser som kräver behörighet \cite{yaws}.

Cowboy är mera likt en applikationsserver som node.js eller tomcat, som är designad med kodexekvering i åtanke. Det är en helt dynamisk applikation där allt generas med kod. Variablerna som tas emot behöver kontrolleras så att dessa inte innehåller skadlig kod
%Hänsyn borde tas på vilka variabler som tas emot, som att kontrollera dessa, så att de inte innehåller skadlig kod 
\cite{cowboy}.

Phoenix är ett helhetswebramverk likt Ruby on Rails, vid start får man en färdig webbapplikation, där utvecklaren skriver kod samt lägger på externa paket
\footnote{Med paket så menas t.ex.
\url{https://hex.pm/packages/cowboy} Eller \url{https://hex.pm/packages/bcrypt} } med kod för att ändra webbapplikationen till att uppfylla målen. En viktig aspekt är även att inte lägga till gamla osäkra paket eller att skriva kod som öppnar säkerhetshål \cite{phoenix/ruby,phoenix/rails,phoenix}.




\subsubsection*{Integritet i Erlang } 
Kommunikation genom meddelanden - Processer i Erlang kommunicerar med varandra genom att skicka meddelanden sinsemellan oberoende av varandras tillstånd.
 Meddelanden som skickas kan lätt taggas, vilket förtydligar vilken process som ska ta emot det.
 Det är inte heller någon data som behöver delas mellan trådar, vilket spar minne.

Isolering - Det enda sättet för en process att kommunicera med omgivningen är via meddelanden.
 I ett system utan sådan egenskap kräver IFC att varje interaktion mellan processer kontrolleras, vilket kan minska prestandan.
 
 Processer är av lättvikt - Processer i Erlang kräver inte lika mycket av minnet i jämförelse med trådar i de mer konventionella programmeringsspråken. Detta är viktigt för IFC: i många applikationer måste en komponent hantera många olika användares konfidentiella data. IFC kräver då att data ska partitioneras till isolerade enheter.
 Att använda konventionella trådar eller processer för att hantera varje sådan enhet ger mycket arbete till ett system. Ska man exempelvis bibehålla utgivarnas integritet krävs det användning av en tråd per utgivare för sändning av meddelanden.
 \cite{IFC}



\section{	Analys och diskussion}


\subsection{Analys}
Eftersom metoden som användes var en litteraturstudie, bygger resultatet på detta arbete på andras arbeten och tester. Även om redan upptäckt kunskap ansammlades, hade resultatet eventuellt blivit mer övertygande om egna tester  hade utförts.
Enligt vår åsikt är dessa webbramverk mer passande för webbapplikationer än andra eftersom deras användande av Erlang/Elixir innebär mer skalbarhet samt att varje process som hanterar känslig data är både isolerad samt att det inte kräver lika mycket minne som trådar och processer gör.

Om man tog samma nödvändiga steg som för andra plattformar och språk, skulle det viktigaste vara att hantera Erlang VM säkert. Skulle en attackerare få åtkomst till en Erlang VM nod skulle denne få tillgång till alla noder i applikationen.
Detta går att hindra genom att använda en proxy \cite{101} server där endast port 80 och port 443 går att ansluta till internet, för att sedan endast öppna en port för nodkommunikation mot ett virtuellt privat nätverk där alla noder ligger.
 Eftersom att ett av de vanligaste hoten idag är DDOS, vilket Erlang VM tillsammans med cloud computing är bättre utrustad än de flesta att hantera.

Alla ramverk har säkerhetsbrister. Exempelvis Erlang-webservern YAWS vars svagheter har kunnat utnyttjas. \cite{exploit} 
	\subsection{Diskussion}

Elixir och Phoenix är relativt nya tekniker\footnote{båda var i version 1 vid skrivandet av rapporten}, de kan växa samt bli väldigt relevanta i framtiden\cite{phoenix/ruby}. Därför är det väldigt fördelaktigt att dessa är säkra från början.

Erlang är bra då man ska implementera säkra databaser för konfidentiell data.
 \cite{database}
    
    
\section{Slutsatser}


Varför Erlang? Erlang har ett procedurellt programmeringsspråk vars modell bygger på att ingen data delas mellan processer, vilket i sin tur gör att den passar för IFC.

Erlang och Elixir-ramverken var säkrare än deras motsvarigheter i Objektorienterade språk. 
Det är upp till utvecklaren att täppa till samma säkerhetshål med samma principer. 
Det som ger Erlang och Elixir en fördel är deras lättviktiga processer, anpassning till cloud computing samt skalbarheten. Eftersom att Erlangs noder  
%Detta gör i sin tur att de 
kan hantera flera användare bättre än deras objektorienterade motsvarigheter och då även hantera DDOS-attacker bättre.

	
\begin{thebibliography}{99}
	
    %1
	\bibitem{erlang}"Erlang" \url{http://www.erlang.org}
hämtad 2016-03-06
	%2
\bibitem{elixir}"Elixir" \url{http://elixir-lang.org}
hämtad 2016-03-06
	%3
\bibitem{owasp} OWASP "OWASP Top 10" \url{https://www.owasp.org/index.php/Top10#OWASP_Top_10_for_2013}
publicerad 2013-06-12,
hämtad 2016-03-06
	%4
    \bibitem{101} {\em Erlang Security 101} by E D Williams, An NCC Group Publication
    %5
\bibitem{IFC} CBCU Eastern Cancer Registry National Health Service, Computer Laboratory University of Cambridge, Department of Computing Imperial College London, {\em “Enforcing User Privacy in Web Applications using Erlang”}, \url{http://w2spconf.com/2010/papers/p09.pdf} Publicerad 2010-05-01.
 Hämtad 2016-03-02.
	%6
				
\bibitem{database}Florian Kammuller Technische Universität Berlin Software Engineering Group flokam@cs.tu-berlin.de, Reiner kammuller, Universität Siegen, Fakultät fur Elektrotechnik und Informatik, reiner.
kammuller@gmail.
com, {\em “Enhancing Privacy Implementations of Database Enquiries”}, ISBN 978-1-4244-3839-6 Publicerad 2009-05-24, Hämtad 2016-03-06

	%7
\bibitem{resource-safe}David Teller Fac.
 des Sci.
, Lab.
 d''Inf.
 Fondamentale d''Orleans, Univ.
 d''Orleans, Bourges {\em “Towards a resource-safe Erlang”} ISBN 978-0-9785699-1-4 Publicerad 2007-05-25, Hämtad 2016-03-06			
	%8
    \bibitem{yawsorg} Claes Wikstrom "Yaws" \url{http://yaws.hyber.org}
hämtad 2016-03-06

\bibitem{cowboyorg}ninenines "Cowboy" \url{https://github.com/ninenines/cowboy}
hämtad 2016-03-06


\bibitem{phoenixorg}  "Phoenix Framework" \url{http://www.phoenixframework.org}
hämtad 2016-03-06

%11

\bibitem{yaws} {\em “Building Web Applications with Erlang by Zachary Kessin (O’Reilly)”}.
 Copyright 2012 Zachary Kessin, 978-1-449-30996-1.

\url{https://www.safaribooksonline.com/u/001o000000lifiuAAA/}



\bibitem{cowboy}Riza Fahmi och Augie De Blieck Jr.
 "Cowboy Tutorial " \url{http://elixirdose.com/post/elixirdose_flatfileblog }
hämtad 2016-03-06

\bibitem{phoenix/ruby} Lau Taarnskov "Elixir - The next big language for the web"\url{http://www.creativedeletion.com/2015/04/19/elixir_next_language.
html}
publicerad  2015-04-19,
hämtad 2016-03-06



\bibitem{phoenix/rails} Szymon Boniecki "Elixir blog in 15 minutes using Phoenix framework" \url{http://codetunes.com/2015/phoenix-blog/}, Jakub Cieślar, publicerad 2015-08-13,
hämtad 2016-03-06

\bibitem{phoenix}Riza Fahmi och Augie De Blieck Jr.
  "Phoenix Flying High" \url{http://elixirdose.com/post/phoenix_part_5_deploying_phoenix_the_final_part} 
hämtad 2016-03-06

%not using





\bibitem{exploit} Michael Brooks "Yaws-Wiki 1.88-1 (Erlang) Stored and Reflective XSS Vulnerabilities"  \url{https://www.exploit-db.com/exploits/17111/}
publicerad 2011-04-04,
hämtad 2016-03-06



\end{thebibliography}



\end{document}
